\documentclass{jsarticle}

\usepackage{amssymb}
\usepackage{graphicx}
\usepackage[dvipdfmx]{color}
\usepackage{here}
\usepackage{tabularx}
\usepackage{amsmath}
\usepackage{url}
\usepackage[hang,small,bf]{caption}
\usepackage[subrefformat=parens]{subcaption}
\usepackage{tikz}
\usepackage{siunitx}
\usepackage{bm}
\usepackage[top=15truemm,bottom=20truemm,left=20truemm,right=20truemm]{geometry}
\usetikzlibrary{shapes.geometric}
\usetikzlibrary {shapes.misc}
\usetikzlibrary{positioning}
\captionsetup{compatibility=false}
 
\begin{document}

日付:4/19

\section*{Ex2.2}

$f:\mathbb{R}^n \mapsto \mathbb{R}$を凸関数とし、$c$を定数とする。
集合$S = \{\bm{x} \in \mathbb{R}^n \mid f(\bm{x})\leq c\}$は凸であることを示せ。
\\
\\
関数$f$は$\mathbb{R}^n$上の凸関数である。任意の$\bm{x},\bm{y}\in S,\lambda \in [0,1]$に対して、
\begin{equation}
f((1-\lambda)\bm{x}+\lambda\bm{y})\leq(1-\lambda)f(\bm{x})+\lambda f(\bm{y})\leq c
\end{equation}
が成り立つ。したがって$(1-\lambda)\bm{x}+\lambda \bm{y} \in S$が成り立つため、集合$S$は凸集合である。


\newpage

\section*{Ex2.8}

標準形の多面体$\{\bm{x}\mid \bm{Ax} = \bm{b},\bm{x}> 0\}$があり、行列$\bm{A}$の行が線形独立であると仮定する。
$\bm{x}$を基底解とし、$J = \{ j \mid x_j \neq 0\}$とする。 
基底により基底解$\bm{x}$が求められることと、すべての列$\bm{A}_j(j\in J)$が基底に含まれることが同値であることを示せ。
\\
\par
まず、p54のProduce for constructing absic solutionより、基底により基底解を求めると、$i \neq B(1),...,B(m)$のとき$x_i=0$とするとあるため、
基底解$\bm{x}_j(j\in J)$の時、列$\bm{A}_j(j\in J)$が基底に含まれることがわかる。

反対に、$\bm{B}=[\bm{A}_j],\bm{x}_B=[x_j](j \in J)$とすると、$\bm{B}\bm{x}_B=\bm{b}$が成り立つため、
列$\bm{A}_j(j \in J)$が基底にある時、基底解$\bm{x}$は$x_j \neq 0 (j \in J)$であることがわかる。

$|J|=m$の時、
\begin{equation}
  \left[\begin{array}{lllllll}
  1 & 1 & 2 & 1 & 0 & 0 & 0 \\
  0 & 1 & 6 & 0 & 1 & 0 & 0 \\
  1 & 0 & 0 & 0 & 0 & 1 & 0 \\
  0 & 1 & 0 & 0 & 0 & 0 & 1
  \end{array}\right] \mathbf{x}=\left[\begin{array}{r}
  8 \\
  12 \\
  4 \\
  6
  \end{array}\right]
\end{equation}
であり、得られている基底解が$\bm{x}=(0,0,4,0,-12,4,6)$の時、$x_j \neq 0$を満たす列は$\{\bm{A}_3,\bm{A}_5,\bm{A}_6,\bm{A}_7\}$であり、
\begin{equation}
  \left[\begin{array}{lllllll}
  2 & 0 & 0 & 0 \\
  6 & 1 & 0 & 0 \\
  0 & 0 & 1 & 0 \\
  0 & 0 & 0 & 1
  \end{array}\right] 
  \left[\begin{array}{r}
    4 \\
    -12 \\
    4 \\
    6
    \end{array}\right]
  =\left[\begin{array}{r}
  8 \\
  12 \\
  4 \\
  6
  \end{array}\right]
\end{equation}
が成り立つ。

$|J|<m$の時、
得られている基底解が$\bm{x}=(4,0,2,0,0,0,6)$の時、$x_j \neq 0$を満たす列は$\{\bm{A}_1,\bm{A}_3,\bm{A}_7\}$であり、
\begin{equation}
  \left[\begin{array}{lllllll}
  1 & 2 & 0 \\
  0 & 6 & 0 \\
  1 & 0 & 0 \\
  0 & 0 & 1
  \end{array}\right] 
  \left[\begin{array}{r}
    4 \\
    2 \\
    6
    \end{array}\right]
  =\left[\begin{array}{r}
  8 \\
  12 \\
  4 \\
  6
  \end{array}\right]
\end{equation}
が成り立つ。また、任意の$\bm{A}_k$をくわえても、
\begin{equation}
  \left[\begin{array}{lllllll}
  1 & 2 & 0 &\mid \\
  0 & 6 & 0 & \bm{A}_k\\
  1 & 0 & 0 &\mid\\
  0 & 0 & 1 &
  \end{array}\right] 
  \left[\begin{array}{r}
    4 \\
    2 \\
    6 \\
    0
    \end{array}\right]
  =\left[\begin{array}{r}
  8 \\
  12 \\
  4 \\
  6
  \end{array}\right]
\end{equation}
が成り立つ。s


\end{document}